\documentclass[journal,12pt,twocolumn]{IEEEtran}

\usepackage{setspace}
\usepackage{gensymb}
\singlespacing
\usepackage[cmex10]{amsmath}

\usepackage{amsthm}

\usepackage{mathrsfs}
\usepackage{txfonts}
\usepackage{stfloats}
\usepackage{bm}
\usepackage{cite}
\usepackage{cases}
\usepackage{subfig}

\usepackage{longtable}
\usepackage{multirow}

\usepackage{enumitem}
\usepackage{mathtools}
\usepackage{steinmetz}
\usepackage{tikz}
\usepackage{circuitikz}
\usepackage{verbatim}
\usepackage{tfrupee}
\usepackage[breaklinks=true]{hyperref}
\usepackage{graphicx}
\usepackage{tkz-euclide}

\usetikzlibrary{calc,math}
\usepackage{listings}
    \usepackage{color}                                            %%
    \usepackage{array}                                            %%
    \usepackage{longtable}                                        %%
    \usepackage{calc}                                             %%
    \usepackage{multirow}                                         %%
    \usepackage{hhline}                                           %%
    \usepackage{ifthen}                                           %%
    \usepackage{lscape}     
\usepackage{multicol}
\usepackage{chngcntr}

\DeclareMathOperator*{\Res}{Res}

\renewcommand\thesection{\arabic{section}}
\renewcommand\thesubsection{\thesection.\arabic{subsection}}
\renewcommand\thesubsubsection{\thesubsection.\arabic{subsubsection}}

\renewcommand\thesectiondis{\arabic{section}}
\renewcommand\thesubsectiondis{\thesectiondis.\arabic{subsection}}
\renewcommand\thesubsubsectiondis{\thesubsectiondis.\arabic{subsubsection}}


\hyphenation{op-tical net-works semi-conduc-tor}
\def\inputGnumericTable{}                                 %%

\lstset{
%language=C,
frame=single, 
breaklines=true,
columns=fullflexible
}
\begin{document}


\newtheorem{theorem}{Theorem}[section]
\newtheorem{problem}{Problem}
\newtheorem{proposition}{Proposition}[section]
\newtheorem{lemma}{Lemma}[section]
\newtheorem{corollary}[theorem]{Corollary}
\newtheorem{example}{Example}[section]
\newtheorem{definition}[problem]{Definition}

\newcommand{\BEQA}{\begin{eqnarray}}
\newcommand{\EEQA}{\end{eqnarray}}
\newcommand{\define}{\stackrel{\triangle}{=}}
\bibliographystyle{IEEEtran}
\raggedbottom
\setlength{\parindent}{0pt}
\providecommand{\mbf}{\mathbf}
\providecommand{\pr}[1]{\ensuremath{\Pr\left(#1\right)}}
\providecommand{\qfunc}[1]{\ensuremath{Q\left(#1\right)}}
\providecommand{\sbrak}[1]{\ensuremath{{}\left[#1\right]}}
\providecommand{\lsbrak}[1]{\ensuremath{{}\left[#1\right.}}
\providecommand{\rsbrak}[1]{\ensuremath{{}\left.#1\right]}}
\providecommand{\brak}[1]{\ensuremath{\left(#1\right)}}
\providecommand{\lbrak}[1]{\ensuremath{\left(#1\right.}}
\providecommand{\rbrak}[1]{\ensuremath{\left.#1\right)}}
\providecommand{\cbrak}[1]{\ensuremath{\left\{#1\right\}}}
\providecommand{\lcbrak}[1]{\ensuremath{\left\{#1\right.}}
\providecommand{\rcbrak}[1]{\ensuremath{\left.#1\right\}}}
\theoremstyle{remark}
\newtheorem{rem}{Remark}
\newcommand{\sgn}{\mathop{\mathrm{sgn}}}
\providecommand{\fourier}{\overset{\mathcal{F}}{ \rightleftharpoons}}
%\providecommand{\hilbert}{\overset{\mathcal{H}}{ \rightleftharpoons}}
\providecommand{\system}{\overset{\mathcal{H}}{ \longleftrightarrow}}
	%\newcommand{\solution}[2]{\textbf{Solution:}{#1}}
\newcommand{\solution}{\noindent \textbf{Solution: }}
\newcommand{\cosec}{\,\text{cosec}\,}
\providecommand{\dec}[2]{\ensuremath{\overset{#1}{\underset{#2}{\gtrless}}}}
\newcommand{\myvec}[1]{\ensuremath{\begin{pmatrix}#1\end{pmatrix}}}
\newcommand{\mydet}[1]{\ensuremath{\begin{vmatrix}#1\end{vmatrix}}}
\numberwithin{equation}{subsection}
\makeatletter
\@addtoreset{figure}{problem}
\makeatother
\let\StandardTheFigure\thefigure
\let\vec\mathbf
\renewcommand{\thefigure}{\theproblem}
\def\putbox#1#2#3{\makebox[0in][l]{\makebox[#1][l]{}\raisebox{\baselineskip}[0in][0in]{\raisebox{#2}[0in][0in]{#3}}}}
     \def\rightbox#1{\makebox[0in][r]{#1}}
     \def\centbox#1{\makebox[0in]{#1}}
     \def\topbox#1{\raisebox{-\baselineskip}[0in][0in]{#1}}
     \def\midbox#1{\raisebox{-0.5\baselineskip}[0in][0in]{#1}}
\vspace{3cm}
\title{Assignment 4}
\author{P Ganesh Nikhil Madhav -CS20BTECH11036}
\maketitle
\newpage
\bigskip
\renewcommand{\thefigure}{\theenumi}
\renewcommand{\thetable}{\theenumi}
Download latex-tikz codes from 
%
\begin{lstlisting}
https://github.com/Nik123-cpp/Assignment-4/blob/main/main.tex
\end{lstlisting}
\section{ Problem UGC|MATH 2019,Q.58}
A sample of size $n =2$ is drawn from a population of size $N=4$ using probability proportional to size without replacement scheme , Where the probabilities proportional to size are

\begin{table}[h!]
\resizebox{\columnwidth}{0.6cm}{%
  \begin{tabular}{|c|c|c|c|c|}
    \hline
    i: & 1 & 2 & 3 & 4\\
    \hline
    $p_{i}$ & 0.4 & 0.2 & 0.2 & 0.2\\
    \hline
  \end{tabular}%
} 
   \caption*{Table : Probability vs Size}
\end{table}  
The probability of inclusion of unit (1) in the sample is 

\begin{enumerate}
\begin{multicols}{4}
\setlength\itemsep{2em}
\item $0.4$
\item $0.6$
\item $0.7$
\item $0.75$
\end{multicols}
\end{enumerate}

\section{Solution }
Let $P_{i}(j)$ represent the probability for selecting unit (j) as second unit after selecting  unit (i) 
\begin{align}
    P_{i}(j)&=\frac{p_{j}}{1-p_{i}}
    \label{eq:eq2}
\end{align}

Let  $\pr{i,j}$ be probability of selecting sample \{i,j\} ,using \eqref{eq:eq2}  is 
\begin{align}
    \pr{i,j}&=P_{i}(j)+P_{j}(i)
    \\
    &=\brak{p_{i}\times \frac{p_{j}}{1-p_{i}}} + \brak{p_{j}\times \frac{p_{i}}{1-p_{j}}}
    \label{eq:eq3}
\end{align}
Total samples(Size $n=2$)are 

\definecolor{green}{RGB}{0 150, 22}
\definecolor{Red}{RGB}{200,60,40}
\definecolor{mycolor}{RGB}{0, 60, 240}
\begin{table}[h!]
\resizebox{\columnwidth}{0.95cm}{%
  \begin{tabular}{|c ||c ||c |c | c| c| c|}
    \hline
    \textcolor{green}{Case }&  \textcolor{Red}{1} & \textcolor{Red}{2} & \textcolor{Red}{3} & \textcolor{Red}{4} & \textcolor{Red}{5} & \textcolor{Red}{6}\\
    \hline
    \textcolor{green}{Sample(size $n=2$)} & \textcolor{mycolor}{\brak{1,2}} & \textcolor{mycolor}{\brak{1,3} }& \textcolor{mycolor}{\brak{1,4} }& \textcolor{mycolor}{\brak{2,3}} & \textcolor{mycolor}{\brak{2,4}}& \textcolor{mycolor}{\brak{3,4}}\\
    \hline
  \end{tabular}%
} 
  \caption{ list of samples}
  \label{tab:label1_test}
\end{table}
Let $P_{i}$ be the probability of inclusion of unit (i) in the sample(size $n=2$),Now i will calculate $P_{1}$ ,Favourable cases for inclusion of unit(1) are case (\textcolor{red}{1,2,3}),So
\begin{align}
    P_{1}&=\pr{1,2} +\pr{1,3}+\pr{1,4}
\end{align}
using \eqref{eq:eq3} and $p_{i}$ from question ,
\begin{align}
    P_{1}&=\frac{7}{30} + \frac{7}{30} + \frac{7}{30}
    \\
    &=0.7
\end{align}
Therefore Option (3) is correct.

\end{document}
